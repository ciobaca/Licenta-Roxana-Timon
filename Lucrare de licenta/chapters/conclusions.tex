\chapter*{Concluzii} 
\addcontentsline{toc}{chapter}{Concluzii}
În cadrul acestei lucrări de licență, am explorat și demonstrat corectitudinea algoritmului de selecție a activităților cu profit maxim, utilizând tehnica de proiectare cunoscută sub numele de programare dinamică. Prin intermediul limbajului imperativ Dafny, am reușit să implementăm algoritmul și să verificăm formal corectitudinea acestuia.

Programarea dinamică s-a dovedit a fi o metodă eficientă pentru rezolvarea problemei de selecție a activităților cu profit maxim, permițându-ne să abordăm problema într-un mod structurat și să optimizăm soluția prin subprobleme suprapuse. Utilizarea Dafny a fost esențială pentru a garanta că implementarea respectă toate cerințele de corectitudine, prin verificarea automată a invarianților și a proprietăților specificate.

Rezultatele obținute au confirmat faptul că algoritmul nu numai că este corect din punct de vedere formal, dar și eficient din punct de vedere al complexității timpului de execuție. Acest lucru demonstrează potențialul puternic al combinării programării dinamice cu verificarea formală folosind limbaje specializate precum Dafny în dezvoltarea algoritmilor corecți și eficienți.

Pe baza acestei lucrări, consider că există oportunități semnificative pentru cercetări viitoare, cum ar fi extinderea tehnicii pentru probleme mai complexe sau adaptarea algoritmului pentru a lucra într-un context paralel sau distribuit. De asemenea, integrarea unor alte limbaje și instrumente de verificare formală ar putea oferi perspective noi și valoroase.

În concluzie, am demonstrat că algoritmul de selecție a activităților poate fi implementat corect și eficient folosind programarea dinamică și verificarea formală în Dafny, contribuind astfel la îmbunătățirea metodelor și tehnologiilor utilizate în dezvoltarea de algoritmi siguri și fiabili.