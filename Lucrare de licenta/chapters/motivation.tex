\chapter*{Motivație} 
\addcontentsline{toc}{chapter}{Motivație}

Am ales să fac această temă deoarece Dafny era un limbaj de programare nou pentru mine si am considerat a fi o provocare. Știam doar că acesta este folosit pentru a asigura o mai mare siguranță si corectitudine, putând fi aplicat în industria aerospațială, în industria medicală, la dezvoltarea sistemelor financiare, în securitate și  criptografie. Totodată, prin demonstrarea corectitudinii unui algoritm in Dafny puteam să-mi folosesc pe langă cunostințele informatice si pe cele  de matematică, de care am fost mereu atrasă. Pentru a crește gradul de complexitate a lucrării am decis să folosesc ca și tehnică de proiectare a aloritmilor programarea dinamică. Astfel, având posibilitatea sa înțeleg mai bine cum funcționează programarea dinamică și să demonstrez că, cu ajutorul ei, se obține o soluție optimă. 