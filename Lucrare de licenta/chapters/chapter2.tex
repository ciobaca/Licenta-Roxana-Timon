\definecolor{codegreen}{rgb}{0,0.6,0}
\definecolor{codegray}{rgb}{0.5,0.5,0.5}
\definecolor{codepurple}{rgb}{0.58,0,0.82}
\definecolor{backcolour}{rgb}{0.95,0.95,0.92}
\lstdefinestyle{mystyle}{
    backgroundcolor=\color{backcolour},   
    commentstyle=\color{codegreen},
    keywordstyle=\color{magenta},
    numberstyle=\tiny\color{codegray},
    stringstyle=\color{codepurple},
    basicstyle=\ttfamily\footnotesize,
    breakatwhitespace=false,         
    breaklines=true,                 
    captionpos=b,                    
    keepspaces=true,                 
    numbers=left,                    
    numbersep=5pt,                  
    showspaces=false,                
    showstringspaces=false,
    showtabs=false,                  
    tabsize=2
}


\lstset{style=mystyle}

\chapter{Programarea dinamică}
Programarea dinamică este o tehnică de proiectare a algoritmilor utilizată pentru rezolvarea problemelor de optimizare.
Pentru a rezolva o anumită problemă folosind programarea dinamică, trebuie să identificăm în mod convenabil mai multe subprobleme.
După ce alegem subproblemele, trebuie să stabilim cum se poate calcula soluția unei subprobleme în funcție de alte subprobleme.
Principala idee din spatele programării dinamice constă în stocarea rezultatelor subproblemele pentru a evita recalcularea lor de fiecare dată când sunt necesare. În general, programarea dinamică se aplică pentru probleme de optimizare pentru care algoritmii greedy nu produc în general soluția optimă. \cite{Curs}

\section{Avantajele programării dinamice față de celelalte tehnici de proiectare}
În ceea ce privește programarea dinamică și tehnica greedy, 
în ambele cazuri apare noțiunea de subproblemă și proprietatea de substructură optimă. De fapt, tehnica greedy poate fi gândită ca un caz particular de programare dinamică, unde rezolvarea unei probleme este determinată direct de alegerea greedy, nefiind nevoie de a
enumera toate alegerile posibile.\cite{Curs} Avantajele programării dinamice fața de tehnica greedy sunt:
\begin{itemize}
  \item \textbf {Optimizare Globală}: : Programarea dinamică are capacitatea de a găsi soluția optimă globală pentru o problemă, în timp ce algoritmii greedy pot fi limitați la luarea deciziilor locale care pot duce la o soluție suboptimală.
  \item \textbf {Flexibilitate}: Programarea dinamică poate fi utilizată pentru o gamă mai largă de probleme, inclusiv cele care implică restricții mai complexe sau soluții care necesită evaluarea mai multor posibilități. În comparație, algoritmii greedy sunt adesea limitați la problemele care pot fi rezolvate prin luarea deciziilor locale în fiecare pas.
  
\end{itemize}
Cu toate acestea, algoritmii greedy pot fi mai potriviți pentru problemele care permit luarea de decizii locale și producerea rapidă a unei soluții aproximative.


\section{Ce este problema de selecție a activităților cu profit maxim ?}
Problema de selecție a activităților cu profit maxim este o problema de optimizare care returneaza pentru o listă de activități distincte (difera prin cel putin un timp) caracterizate prin timp de început, timp de încheiere și profit,  ordonate dupa timpul de încheiere, o secventă de activități care nu se suprapun doua cate doua și care produc un profit maxim. 
\begin{lstlisting}[numbers=None]
    Input: O secventa de activitati caracterizate prin
    { timp de inceput, timp de incheiere,  profit}
       Activitate 1:  {1, 2, 50} 
       Activitate 2:  {3, 5, 20}
       Activitate 3:  {6, 19, 100}
       Activitate 4:  {2, 100, 200}
    Output: Profit-ul maxim este 250, pentru solutia optima formata din activitatea 1 si activitatea 4.
\end{lstlisting}

\section{Pseudocod}
\begin{lstlisting}[language=C++]

struct Activitate {
    timp de inceput: int
    timp de incheiere : int
    profit : int 
}
function planificareActivitatiPonderate(activitati):
    // activitatile sunt sortate crescator dupa timpul de incheiere
    // initializam un vector pentru a stoca profiturile maxime pentru fiecare activitate, fie dp un vector de dimensiune n, unde n este numarul de activitati
    dp[0] = activitati[0].profit
    solutie[0] = [activitati[0]] //un vector binar 0 - nu am ales activitatea si 1 - am ales activitatea
    solutiiOptime = solutie //stocam solutiile optime la fiecare pas
    // Programare dinamica pentru a gasi profitul maxim
    pentru i de la 1 la n-1:
        // Gaseste cea mai recenta activitate care nu intra in conflict cu activitatea curenta
        solutiaCuActivitateaI = [1] // selectam activitatea curenta
        ultimaActivitateNeconflictuala = gasesteUltimaActivitateNeconflictuala(activitati, i)
        
        // Calculează profitul maxim incluzand activitatea curenta si excluzand-o
        profitulCuActivitateaCurenta = activitati[i].profit
        daca ultimaActivitateNeconflictuala != -1:
            profitulIncluderiiActivitatiiCurente += dp[ultimaActivitateNeconflictuala]
            solutiaCuActivitateaI = solutiiOptime[ultimaActivitateNeconflictuala] + solutiaCuActivitateaI
        dp[i] = maxim(profitulIncluderiiActivitatiiCurente, dp[i-1])
        
        daca profitulIncluderiiActivitatiiCurente > dp[i-1]:
            solutie = solutieCuActivitateaI
        else
            solutie = solutie + [0]
        solutiiOptime = solutiiOptime + solutie 
    // Returneaza solutia si profitul maxim
    return solutie, dp[n-1]

function gasesteUltimaActivitateNeconflictuala(activitati, indexCurent):
    pentru j de la indexCurent-1 la 0:
        daca activitati[j].timpDeIncheiere <= activitati[indexCurent].timpDeInceput:
            return j
    return -1
\end{lstlisting}

\section{Cum funcționeaza programarea dinamică în cazul problemei de selectie a activitatilor}
Pentru problema de selectie a activitatilor se poate folosi programarea dinamică, deoarece aceasta poate fi împarțită în subprobleme, respectiv la fiecare pas putem forma o soluție parțiala optimă, de al carei profit ne putem folosi la urmatorul pas. \\


\subsection{Solutia partiala si solutia partiala optima}
O solutie partiala trebuie sa contina activitati care nu se suprapun si sa aiba o lungime prestabilita. O solutie partiala este si optima, daca profitul acesteia este mai mare sau egal decat al oricarei alte solutii partiale de aceeasi lungime.  \\

Pentru datele de intrare precizate de mai sus se obtin urmatoarele valori:\\
\textbf{La primul pas}:
\begin{enumerate}
    \item soluția parțiala optimă este formata din Activitatea 1.
    \item iar profit-ul maxim este 50. \\
\end{enumerate}
\textbf{La al 2-lea pas}:
\begin{enumerate}
    \item deoarece Activitatea 1 nu se suprapune cu Activitatea 2 se obține soluția partiala optimă formata din Activitatea 1 și Activitatea 2.
    \item profitul optim la pasul 2 este 50 + 20 = 70, care este mai mare decat cel anterior (condiție necesară).
\end{enumerate}
\textbf{La al 3-lea pas}:
\begin{enumerate}
    \item soluția parțială optimă este formata din Activitatea 1, Activitatea 2 și Activitatea 3, deoarece Activitatea 3 nu se suprapune cu activitatea 2 ( înseamnă că nu se suprapune cu soluția partiala de la al 2-lea pas, fiind ordonate dupa timpul se sfârsit) si le putem concatena.
    \item profitul pentru această soluție parțiala este strict mai mare decat cel de la pasul 2, noul profit optim devenind 170. \\
\end{enumerate}
\textbf{La al 4-lea pas}:
\begin{enumerate}
    \item soluția parțială ce conține Activitatea 4 este formata din Activitatea 4 si Activitatea 1.
    \item profitul pentru aceasta soluție parțiala este strict mai mare decat profitul optim anterior = 170, noul profit optim devenind 250.
\end{enumerate} 
Astfel, soluția problemei este cea de la ultimul pas, fiind formata din Activitatea 1 și Activitatea 4 si avand profitul optim 250. 

\subsection{Proprietatea de substructura optimă în cazul problemei de selecție a activităților}
În cazul problemei de selecție a activităților, proprietatea de substructură optimă este identificată atunci cand eliminăm o activitate dintr-o soluție parțială optimă si obținem tot o soluție parțială optimă. Altfel, dacă acest proprietate nu ar fi valabilă, înseamna ca soluția parțială optimă pe care o descompunem nu este cu adevărat optimă. 